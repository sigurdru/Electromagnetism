%%
%% Automatically generated file from DocOnce source
%% (https://github.com/hplgit/doconce/)
%%
%%
% #ifdef PTEX2TEX_EXPLANATION
%%
%% The file follows the ptex2tex extended LaTeX format, see
%% ptex2tex: http://code.google.com/p/ptex2tex/
%%
%% Run
%%      ptex2tex myfile
%% or
%%      doconce ptex2tex myfile
%%
%% to turn myfile.p.tex into an ordinary LaTeX file myfile.tex.
%% (The ptex2tex program: http://code.google.com/p/ptex2tex)
%% Many preprocess options can be added to ptex2tex or doconce ptex2tex
%%
%%      ptex2tex -DMINTED myfile
%%      doconce ptex2tex myfile envir=minted
%%
%% ptex2tex will typeset code environments according to a global or local
%% .ptex2tex.cfg configure file. doconce ptex2tex will typeset code
%% according to options on the command line (just type doconce ptex2tex to
%% see examples). If doconce ptex2tex has envir=minted, it enables the
%% minted style without needing -DMINTED.
% #endif

% #define PREAMBLE

% #ifdef PREAMBLE
%-------------------- begin preamble ----------------------

\documentclass[%
oneside,                 % oneside: electronic viewing, twoside: printing
final,                   % draft: marks overfull hboxes, figures with paths
10pt]{article}

\listfiles               %  print all files needed to compile this document

\usepackage{relsize,makeidx,color,setspace,amsmath,amsfonts,amssymb}
\usepackage[table]{xcolor}
\usepackage{bm,ltablex,microtype}

\usepackage[pdftex]{graphicx}

\usepackage[T1]{fontenc}
%\usepackage[latin1]{inputenc}
\usepackage{ucs}
\usepackage[utf8x]{inputenc}

\usepackage{lmodern}         % Latin Modern fonts derived from Computer Modern

% Hyperlinks in PDF:
\definecolor{linkcolor}{rgb}{0,0,0.4}
\usepackage{hyperref}
\hypersetup{
    breaklinks=true,
    colorlinks=true,
    linkcolor=linkcolor,
    urlcolor=linkcolor,
    citecolor=black,
    filecolor=black,
    %filecolor=blue,
    pdfmenubar=true,
    pdftoolbar=true,
    bookmarksdepth=3   % Uncomment (and tweak) for PDF bookmarks with more levels than the TOC
    }
%\hyperbaseurl{}   % hyperlinks are relative to this root

\setcounter{tocdepth}{2}  % levels in table of contents

% prevent orhpans and widows
\clubpenalty = 10000
\widowpenalty = 10000

\newenvironment{doconceexercise}{}{}
\newcounter{doconceexercisecounter}


% ------ header in subexercises ------
%\newcommand{\subex}[1]{\paragraph{#1}}
%\newcommand{\subex}[1]{\par\vspace{1.7mm}\noindent{\bf #1}\ \ }
\makeatletter
% 1.5ex is the spacing above the header, 0.5em the spacing after subex title
\newcommand\subex{\@startsection{paragraph}{4}{\z@}%
                  {1.5ex\@plus1ex \@minus.2ex}%
                  {-0.5em}%
                  {\normalfont\normalsize\bfseries}}
\makeatother


% --- end of standard preamble for documents ---


% insert custom LaTeX commands...

\raggedbottom
\makeindex
\usepackage[totoc]{idxlayout}   % for index in the toc
\usepackage[nottoc]{tocbibind}  % for references/bibliography in the toc

%-------------------- end preamble ----------------------

\begin{document}

% matching end for #ifdef PREAMBLE
% #endif

\newcommand{\exercisesection}[1]{\subsection*{#1}}


% ------------------- main content ----------------------



% --- begin exercise ---
\begin{doconceexercise}
\refstepcounter{doconceexercisecounter}

\exercisesection{Exercise \thedoconceexercisecounter: Two particles}


In this exercise we are going to animate the movement of two charged particles.


\subex{a)}
Write down Coulomb's law.


% --- begin solution of exercise ---
\paragraph{Solution.}
\begin{equation}
\mathbf{F} = \frac{1}{4\pi \epsilon _0} \frac{q_1q_2}{r²}\hat{\mathbf{r}}
\end{equation}

% --- end solution of exercise ---

\subex{b)}
Code a function that takes the postions and charges of two particles, and then returns the force between them.


% --- begin solution of exercise ---
\paragraph{Solution.}
\begin{verbatim}
def F(q1, q2, r1, r2):
    """
    This function takes in the positions and charges of two particles,
    and returns the force.
    """
    r21 = r1 - r2
    R21 = np.linalg.norm(r21, axis=0)
    F21 = 1/(4*np.pi*eps0)*q1*q2*r21/R21**3
    return F21
\end{verbatim}

% --- end solution of exercise ---

\subex{c)}
Using your numeric method of choice, simulete how the particles interact. It can be a good idea to set $\epsilon _0=1$. Try for different initial conditions, plot the result and see if it makes sence.

% --- begin hint in exercise ---

\paragraph{Hint.}
You can try the initial conditions:
\begin{verbatim}
r0_1 = np.array([0, 0])
r0_2 = np.array([-1, -1])
v0_1 = np.array([0, 0])
v0_2 = np.array([2, 1.5])
Q1 = 1
Q2 = 3
m1 = 0.2
m2 = 0.1
eps0 = 1
dt = 1e-4
tstart = 0
tslutt = 2
\end{verbatim}

% --- end hint in exercise ---


% --- begin solution of exercise ---
\paragraph{Solution.}
Expanding on the code over, I wrote the simulation using Euler-Chromer.
\begin{verbatim}
#Define values we are going to use
r0_1 = np.array([0, 0])
r0_2 = np.array([-1, -1])
v0_1 = np.array([0, 0])
v0_2 = np.array([2, 1.5])
Q1 = 1
Q2 = 3
m1 = 0.2
m2 = 0.1
eps0 = 1
dt = 1e-4
tstart = 0
tslutt = 2
N = int((tslutt - tstart)/dt)
#Define som arrays we are going to fill
pos1 = np.zeros((2, N))
pos2 = np.zeros((2, N))
vel1 = np.zeros((2, N))
vel2 = np.zeros((2, N))
#Set initial conditions
pos1[:, 0] = r0_1
pos2[:, 0] = r0_2
vel1[:, 0] = v0_1
vel2[:, 0] = v0_2
for i in range(N-1):
    """
    Using Euler-Chromer
    """
    a1 = F(Q1, Q2, pos1[:, i], pos2[:, i])/m1
    a2 = -a1*m1/m2
    vel1[:, i+1] = a1*dt + vel1[:, i]
    vel2[:, i+1] = a2*dt + vel2[:, i]
    pos1[:, i+1] = vel1[:, i+1]*dt + pos1[:, i]
    pos2[:, i+1] = vel2[:, i+1]*dt + pos2[:, i]
plt.plot(pos1[0], pos1[1], color='r', \label='particle1')
plt.plot(pos2[0], pos2[1], color='b', \label='particle2')
plt.x\label('Posisjon i x-retning')
plt.y\label('Posision i y-retning')
plt.legend(loc=1)
plt.show()
\end{verbatim}

% --- end solution of exercise ---

\subex{d)}
Now that you know the positions it can be a good idea to animate the particles to get a better look at the result. Animate the motion of the particles, does the animation agree with your expectations? Try for different initial conditions.

% --- begin hint in exercise ---

\paragraph{Hint.}
You can use matplotlib to plot the animation.
\begin{verbatim}
from matplotlib import animation
\end{verbatim}

% --- end hint in exercise ---


% --- begin solution of exercise ---
\paragraph{Solution.}
Expanding the code we have written so far.
\begin{verbatim}
#Creating the figure we are going to use
fig = plt.figure()
ax = plt.axes(xlim=(-2, 2), ylim=(-2, 2))
#The two particles
particle1, = ax.plot([], [], 'ro', \label='Particle1')
particle2, = ax.plot([], [], 'bo', \label='Particle2')
points = [particle1, particle2]
def init():
    #Init function
    ax.clear
    return points
#This adjusts the speed of the animation
speed = 20
def animate(i):
    #The function that animates
    ax.set_title('Tid = %f' % (i*dt*speed))
    points[0].set_data(pos1[:, i*speed])
    points[1].set_data(pos2[:, i*speed])
    return points
#Plotting
anim = animation.FuncAnimation(fig, animate, init_func=init,
                               frames=int(N/speed), interval=1, blit=False)
plt.x\label('Position in x-direction')
plt.y\label('Position in y-direction')
plt.legend(loc=1)
plt.show()
\end{verbatim}

% --- end solution of exercise ---

\end{doconceexercise}
% --- end exercise ---


% ------------------- end of main content ---------------

% #ifdef PREAMBLE
\end{document}
% #endif

