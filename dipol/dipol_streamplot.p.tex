%%
%% Automatically generated file from DocOnce source
%% (https://github.com/hplgit/doconce/)
%%
%%
% #ifdef PTEX2TEX_EXPLANATION
%%
%% The file follows the ptex2tex extended LaTeX format, see
%% ptex2tex: http://code.google.com/p/ptex2tex/
%%
%% Run
%%      ptex2tex myfile
%% or
%%      doconce ptex2tex myfile
%%
%% to turn myfile.p.tex into an ordinary LaTeX file myfile.tex.
%% (The ptex2tex program: http://code.google.com/p/ptex2tex)
%% Many preprocess options can be added to ptex2tex or doconce ptex2tex
%%
%%      ptex2tex -DMINTED myfile
%%      doconce ptex2tex myfile envir=minted
%%
%% ptex2tex will typeset code environments according to a global or local
%% .ptex2tex.cfg configure file. doconce ptex2tex will typeset code
%% according to options on the command line (just type doconce ptex2tex to
%% see examples). If doconce ptex2tex has envir=minted, it enables the
%% minted style without needing -DMINTED.
% #endif

% #define PREAMBLE

% #ifdef PREAMBLE
%-------------------- begin preamble ----------------------

\documentclass[%
oneside,                 % oneside: electronic viewing, twoside: printing
final,                   % draft: marks overfull hboxes, figures with paths
10pt]{article}

\listfiles               %  print all files needed to compile this document

\usepackage{relsize,makeidx,color,setspace,amsmath,amsfonts,amssymb}
\usepackage[table]{xcolor}
\usepackage{bm,ltablex,microtype}

\usepackage[pdftex]{graphicx}

\usepackage[T1]{fontenc}
%\usepackage[latin1]{inputenc}
\usepackage{ucs}
\usepackage[utf8x]{inputenc}

\usepackage{lmodern}         % Latin Modern fonts derived from Computer Modern

% Hyperlinks in PDF:
\definecolor{linkcolor}{rgb}{0,0,0.4}
\usepackage{hyperref}
\hypersetup{
    breaklinks=true,
    colorlinks=true,
    linkcolor=linkcolor,
    urlcolor=linkcolor,
    citecolor=black,
    filecolor=black,
    %filecolor=blue,
    pdfmenubar=true,
    pdftoolbar=true,
    bookmarksdepth=3   % Uncomment (and tweak) for PDF bookmarks with more levels than the TOC
    }
%\hyperbaseurl{}   % hyperlinks are relative to this root

\setcounter{tocdepth}{2}  % levels in table of contents

% prevent orhpans and widows
\clubpenalty = 10000
\widowpenalty = 10000

\newenvironment{doconceexercise}{}{}
\newcounter{doconceexercisecounter}


% ------ header in subexercises ------
%\newcommand{\subex}[1]{\paragraph{#1}}
%\newcommand{\subex}[1]{\par\vspace{1.7mm}\noindent{\bf #1}\ \ }
\makeatletter
% 1.5ex is the spacing above the header, 0.5em the spacing after subex title
\newcommand\subex{\@startsection{paragraph}{4}{\z@}%
                  {1.5ex\@plus1ex \@minus.2ex}%
                  {-0.5em}%
                  {\normalfont\normalsize\bfseries}}
\makeatother


% --- end of standard preamble for documents ---


% insert custom LaTeX commands...

\raggedbottom
\makeindex
\usepackage[totoc]{idxlayout}   % for index in the toc
\usepackage[nottoc]{tocbibind}  % for references/bibliography in the toc

%-------------------- end preamble ----------------------

\begin{document}

% matching end for #ifdef PREAMBLE
% #endif

\newcommand{\exercisesection}[1]{\subsection*{#1}}


% ------------------- main content ----------------------



% --- begin exercise ---
\begin{doconceexercise}
\refstepcounter{doconceexercisecounter}

\exercisesection{Exercise \thedoconceexercisecounter: Visualize the field from a dipole}


\emph{(Made by Sigurd Sørlie Rustad)}

\noindent
In this exercise we are going to visualize the field from a magnetic dipole with a streamplot. The field from a dipole is given by the equation
\begin{equation}
\mathbf{B}(\mathbf{r}) = \frac{\mu_0}{4\pi}\bigg(\frac{3\mathbf{r}(\mathbf{m}\cdot\mathbf{r})}{r⁵}-\frac{\mathbf{m}}{r³}\bigg)
\end{equation}
Make a function that takes magnetic moment $\mathbf{m}$, its location and the position $\mathbf{r}$ you want to evaluate the field.
The output should be the resulting magnetic field. You only need to do it in 2D. Make a streamplot to visualize the field.

\noindent
Extra challenge: Try to vectorize your code. If you do it properly it will be compatible with both 2D and 3D. NumPy has several
good packages that you can use to vectorize your code. The ones used in the solution are tensordot to dotproduct along an axis
and linalg.norm to find the length of several vectors along an axis.


% --- begin solution of exercise ---
\paragraph{Solution.}
\begin{verbatim}
import numpy as np
import matplotlib.pyplot as plt
import scipy.constants as const
mu0 = const.mu_0
def dipole(m, r, r0):
    """
    This function take the magnetic dipole m, with position r0, 
    and returnsthe magnetic field in postion r.
    """
    #np.subtract allows r0 to be a list and not an array
    R = np.subtract(np.transpose(r), r0).T
    #Finding the length of the vector
    norm_R = np.linalg.norm(R, axis=0)
    #Tensordot allows us to dotproduct along an axis
    m_dot_R = np.tensordot(m, R, axes=1)
    #calculatig the magnetic field
    B = 3*m_dot_R*R/norm_R**5 - np.tensordot(m, 1/norm_R**3, axes=0)
    #multiplying with the physical constant
    B *= mu0/(4*np.pi)
    return B
    
xa = np.linspace(-1, 1)
ya = np.linspace(-1, 1)
Bx, By = dipole(m = [0, 0.1], r = np.meshgrid(xa, ya),\
     r0=[-0.2, 0.1])
plt.streamplot(xa, ya, Bx, By)
plt.show()
\end{verbatim}

% --- end solution of exercise ---

\end{doconceexercise}
% --- end exercise ---


% ------------------- end of main content ---------------

% #ifdef PREAMBLE
\end{document}
% #endif

