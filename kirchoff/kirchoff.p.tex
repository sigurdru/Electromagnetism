%%
%% Automatically generated file from DocOnce source
%% (https://github.com/hplgit/doconce/)
%%
%%
% #ifdef PTEX2TEX_EXPLANATION
%%
%% The file follows the ptex2tex extended LaTeX format, see
%% ptex2tex: http://code.google.com/p/ptex2tex/
%%
%% Run
%%      ptex2tex myfile
%% or
%%      doconce ptex2tex myfile
%%
%% to turn myfile.p.tex into an ordinary LaTeX file myfile.tex.
%% (The ptex2tex program: http://code.google.com/p/ptex2tex)
%% Many preprocess options can be added to ptex2tex or doconce ptex2tex
%%
%%      ptex2tex -DMINTED myfile
%%      doconce ptex2tex myfile envir=minted
%%
%% ptex2tex will typeset code environments according to a global or local
%% .ptex2tex.cfg configure file. doconce ptex2tex will typeset code
%% according to options on the command line (just type doconce ptex2tex to
%% see examples). If doconce ptex2tex has envir=minted, it enables the
%% minted style without needing -DMINTED.
% #endif

% #define PREAMBLE

% #ifdef PREAMBLE
%-------------------- begin preamble ----------------------

\documentclass[%
oneside,                 % oneside: electronic viewing, twoside: printing
final,                   % draft: marks overfull hboxes, figures with paths
10pt]{article}

\listfiles               %  print all files needed to compile this document

\usepackage{relsize,makeidx,color,setspace,amsmath,amsfonts,amssymb}
\usepackage[table]{xcolor}
\usepackage{bm,ltablex,microtype}

\usepackage[pdftex]{graphicx}

\usepackage[T1]{fontenc}
%\usepackage[latin1]{inputenc}
\usepackage{ucs}
\usepackage[utf8x]{inputenc}

\usepackage{lmodern}         % Latin Modern fonts derived from Computer Modern

% Hyperlinks in PDF:
\definecolor{linkcolor}{rgb}{0,0,0.4}
\usepackage{hyperref}
\hypersetup{
    breaklinks=true,
    colorlinks=true,
    linkcolor=linkcolor,
    urlcolor=linkcolor,
    citecolor=black,
    filecolor=black,
    %filecolor=blue,
    pdfmenubar=true,
    pdftoolbar=true,
    bookmarksdepth=3   % Uncomment (and tweak) for PDF bookmarks with more levels than the TOC
    }
%\hyperbaseurl{}   % hyperlinks are relative to this root

\setcounter{tocdepth}{2}  % levels in table of contents

% prevent orhpans and widows
\clubpenalty = 10000
\widowpenalty = 10000

\newenvironment{doconceexercise}{}{}
\newcounter{doconceexercisecounter}


% ------ header in subexercises ------
%\newcommand{\subex}[1]{\paragraph{#1}}
%\newcommand{\subex}[1]{\par\vspace{1.7mm}\noindent{\bf #1}\ \ }
\makeatletter
% 1.5ex is the spacing above the header, 0.5em the spacing after subex title
\newcommand\subex{\@startsection{paragraph}{4}{\z@}%
                  {1.5ex\@plus1ex \@minus.2ex}%
                  {-0.5em}%
                  {\normalfont\normalsize\bfseries}}
\makeatother


% --- end of standard preamble for documents ---


% insert custom LaTeX commands...

\raggedbottom
\makeindex
\usepackage[totoc]{idxlayout}   % for index in the toc
\usepackage[nottoc]{tocbibind}  % for references/bibliography in the toc

%-------------------- end preamble ----------------------

\begin{document}

% matching end for #ifdef PREAMBLE
% #endif

\newcommand{\exercisesection}[1]{\subsection*{#1}}


% ------------------- main content ----------------------



% --- begin exercise ---
\begin{doconceexercise}
\refstepcounter{doconceexercisecounter}

\exercisesection{Exercise \thedoconceexercisecounter: Kirchoff's circuit laws}


In this exercise we are going to look at Kirchoff's laws, and solve problems using linear algebra.


\subex{a)}
Write down Kirchoff's laws.


% --- begin solution of exercise ---
\paragraph{Solution.}
1. The sum of voltages around any loop is zero. \\
2. The sum of currents at any junction is zero.

% --- end solution of exercise ---

\subex{b)}
The voltages are $U_1 = 10\rm{V}$ and $U_2 = 5\rm{V}$. The resistances are $R_1 = 10\Omega$, $R_2 = 20\Omega$ and $R_3 = 30\Omega$. Using Kirchoff's laws, find the equations needed to solve for the currents. Write it as a matrix equation and use Python to inverse the matrix and find the solution.
\begin{figure}[h]
\centering
\includegraphics[width=0.8\textwidth]{krets1.png}
\end{figure}


% --- begin solution of exercise ---
\paragraph{Solution.}
Using the first law on the left-hand loop we get
\begin{equation}
U_1 -I_1R_1 - I_3R_3 = 0
\end{equation}
Right-hand loop gives
\begin{equation}
U_2 + I_2R_2 - I_3R_3 = 0
\end{equation}
Using the second law on the top center junction, we get
\begin{equation}
I_1 -I_2 - I_3 = 0
\end{equation}
We can write the equations as a matrix equation on the form
\begin{equation}
\left( \begin{matrix} R_1 & 0 & R_3 \\ 0 & -R_2 & R_3 \\ 1 & -1 & -1 \end{matrix}\right) \left(\begin{matrix}I_1 \\ I_2 \\ I_3\end{matrix}\right) =\left(\begin{matrix}U_1 \\ U_2 \\ 0\end{matrix}\right)
\end{equation}
Using the following code
\begin{verbatim}
import numpy as np
R1 = 10  # [ohm]
R2 = 20  # [ohm]
R3 = 30  # [ohm]
U1 = 10  # [V]
U2 = 5   # [V]
A = np.array([[R1, 0, R3],[0, -R2, R3],[1, -1, -1]])
#inverting the matrix A
Ainv = np.linalg.inv(A)
B = np.array([10, 5, 0])
#finding the solution
C = np.dot(Ainv,B)
print(C)
\end{verbatim}
We found the solutions $I_1 \approx 0.318\rm{A}$, $I_2 \approx 0.091\rm{A}$ and $I_3 \approx 0.227\rm{A}$.

% --- end solution of exercise ---

\subex{c)}
Do the same for the circuit under as you did for the previous exersice. Use $R_4 = 40\Omega$, $R_5 = 50\Omega$, $R_6 = 60\Omega$, $U_3 = 12\rm{V}$ and $U_4 = 24\rm{V}$. 
\begin{figure}[h]
\centering
\includegraphics[width=0.8\textwidth]{krets2.png}
\end{figure}


% --- begin solution of exercise ---
\paragraph{Solution.}
Notice that we can reuse the equations over, so we only need to find three more equations. Using the first law on the bottom left loop we get
\begin{equation}
U_4 + I_4R_4 = 0
\end{equation}
Bottom center loop
\begin{equation}
I_4R_4 - I_5R_5 = 0
\end{equation}
Right loop
\begin{equation}
U_2 - U_3 - I_6R_6 + I_5R_5 = 0
\end{equation}
We therefore get the matrix equation
\begin{equation}
\left( \begin{matrix} R_1 & 0 & R_3 & 0 & 0 & 0 \\ 0 & -R_2 & R_3 & 0 & 0 & 0 \\ 1 & -1 & -1 & 0 & 0 & 0 \\ 0 & 0 & 0 & -R_4 & 0 & 0 \\ 0 & 0 & 0 & R_4 & -R_5 & 0\\ 0 & 0 & 0 & 0 & -R_5 & R_6 \end{matrix}\right) \left(\begin{matrix}I_1 \\ I_2 \\ I_3 \\ I_4 \\ I_5 \\ I_6\end{matrix}\right) =\left(\begin{matrix}U_1 \\ U_2 \\ 0 \\ U_4 \\ 0 \\ U_2 - U_3 \end{matrix}\right)
\end{equation}
Using the code
\begin{verbatim}
import numpy as np
R1 = 10  # [ohm]
R2 = 20  # [ohm]
R3 = 30  # [ohm]
R4 = 40  # [ohm]
R5 = 50  # [ohm]
R6 = 60  # [ohm]
U1 = 10  # [V]
U2 = 5   # [V]
U3 = 12  # [V]
U4 = 24   # [V]
A = np.array([[R1, 0, R3, 0, 0, 0], 
              [0, -R2, R3, 0, 0, 0], 
              [1, -1, -1, 0, 0, 0],
              [0, 0, 0,-R4, 0, 0], 
              [0, 0, 0, R4, -R5, 0], 
              [0, 0, 0, 0, -R5, R6]])
#inverting the matrix A
Ainv = np.linalg.inv(A)
B = np.array([U1, U2, 0, U4, 0, U2-U3])
#finding the solution
C = np.dot(Ainv, B)
print(C)
\end{verbatim}
We get the solutions $I_1 \approx 0.318\rm{A}$, $I_2 \approx 0.091\rm{A}$, $I_3 \approx 0.227\rm{A}$, $I_4 = -0.6\rm{A}$, $I_5 = -0.48\rm{A}$ and $I_6 \approx -0.517\rm{A}$.

% --- end solution of exercise ---

\end{doconceexercise}
% --- end exercise ---


% ------------------- end of main content ---------------

% #ifdef PREAMBLE
\end{document}
% #endif

