%%
%% Automatically generated file from DocOnce source
%% (https://github.com/hplgit/doconce/)
%%
%%
% #ifdef PTEX2TEX_EXPLANATION
%%
%% The file follows the ptex2tex extended LaTeX format, see
%% ptex2tex: http://code.google.com/p/ptex2tex/
%%
%% Run
%%      ptex2tex myfile
%% or
%%      doconce ptex2tex myfile
%%
%% to turn myfile.p.tex into an ordinary LaTeX file myfile.tex.
%% (The ptex2tex program: http://code.google.com/p/ptex2tex)
%% Many preprocess options can be added to ptex2tex or doconce ptex2tex
%%
%%      ptex2tex -DMINTED myfile
%%      doconce ptex2tex myfile envir=minted
%%
%% ptex2tex will typeset code environments according to a global or local
%% .ptex2tex.cfg configure file. doconce ptex2tex will typeset code
%% according to options on the command line (just type doconce ptex2tex to
%% see examples). If doconce ptex2tex has envir=minted, it enables the
%% minted style without needing -DMINTED.
% #endif

% #define PREAMBLE

% #ifdef PREAMBLE
%-------------------- begin preamble ----------------------

\documentclass[%
oneside,                 % oneside: electronic viewing, twoside: printing
final,                   % draft: marks overfull hboxes, figures with paths
10pt]{article}

\listfiles               %  print all files needed to compile this document

\usepackage{relsize,makeidx,color,setspace,amsmath,amsfonts,amssymb}
\usepackage[table]{xcolor}
\usepackage{bm,ltablex,microtype}

\usepackage[pdftex]{graphicx}

\usepackage[T1]{fontenc}
%\usepackage[latin1]{inputenc}
\usepackage{ucs}
\usepackage[utf8x]{inputenc}

\usepackage{lmodern}         % Latin Modern fonts derived from Computer Modern

% Hyperlinks in PDF:
\definecolor{linkcolor}{rgb}{0,0,0.4}
\usepackage{hyperref}
\hypersetup{
    breaklinks=true,
    colorlinks=true,
    linkcolor=linkcolor,
    urlcolor=linkcolor,
    citecolor=black,
    filecolor=black,
    %filecolor=blue,
    pdfmenubar=true,
    pdftoolbar=true,
    bookmarksdepth=3   % Uncomment (and tweak) for PDF bookmarks with more levels than the TOC
    }
%\hyperbaseurl{}   % hyperlinks are relative to this root

\setcounter{tocdepth}{2}  % levels in table of contents

% prevent orhpans and widows
\clubpenalty = 10000
\widowpenalty = 10000

\newenvironment{doconceexercise}{}{}
\newcounter{doconceexercisecounter}


% ------ header in subexercises ------
%\newcommand{\subex}[1]{\paragraph{#1}}
%\newcommand{\subex}[1]{\par\vspace{1.7mm}\noindent{\bf #1}\ \ }
\makeatletter
% 1.5ex is the spacing above the header, 0.5em the spacing after subex title
\newcommand\subex{\@startsection{paragraph}{4}{\z@}%
                  {1.5ex\@plus1ex \@minus.2ex}%
                  {-0.5em}%
                  {\normalfont\normalsize\bfseries}}
\makeatother


% --- end of standard preamble for documents ---


% insert custom LaTeX commands...

\raggedbottom
\makeindex
\usepackage[totoc]{idxlayout}   % for index in the toc
\usepackage[nottoc]{tocbibind}  % for references/bibliography in the toc

%-------------------- end preamble ----------------------

\begin{document}

% matching end for #ifdef PREAMBLE
% #endif

\newcommand{\exercisesection}[1]{\subsection*{#1}}


% ------------------- main content ----------------------



% --- begin exercise ---
\begin{doconceexercise}
\refstepcounter{doconceexercisecounter}

\exercisesection{Exercise \thedoconceexercisecounter: Particles in a box}


\emph{(Made by: Sigurd Sørlie Rustad)}

\noindent
In this exercise we are going to simulate and animate particles in a box. We are going to do this in steps, so that the process becomes easier, but also to verify that our results makes sence. We will work with electrons inside a square box with sides $L$. Although we will be working in two dimensions, if you vectorize your code, expanding to 3D is quite straight forward.


\subex{a)}
Create a class \texttt{box}, and a \texttt{\_\_init\_\_} function that takes the amount of particles $N$, length of the side of the box $L$, the charge of the particles $q$, mass of the particles $m$ and vacuum permeability $\epsilon_0$. You also need to place the particles in the box, select their positions randomly inside the box with velocities selected from a normal distribution.

% --- begin hint in exercise ---

\paragraph{Hint.}
To place the particles inside the box you can use
\begin{verbatim}
import numpy as np
r = np.random.uniform(low, high, size)
v = np.random.uniform(loc, scale, size)
\end{verbatim}

% --- end hint in exercise ---


% --- begin solution of exercise ---
\paragraph{Solution.}
\begin{verbatim}
class box:
    def __init__(self, N, L, q, m, eps0):
        """
        Set initial conditions
        """
        self.N = int(N)                     # num of particles
        self.L = L                          # length of side
        self.q = q                          # charge
        self.eps0 = eps0                    # permittivity of vacuum
        self.m = m                          # mass
        self.r = np.random.uniform(
            low=-L/2, high=L/2, size=(2, self.N)
        )                                   # initial positions
        self.v = np.random.normal(
            loc=0, scale=10, size=(2, self.N)
        )                                   # initial velocities
\end{verbatim}

% --- end solution of exercise ---

\subex{b)}
Now we want to create a function \texttt{force} that takes the current positions and returns the force on each particle. We are only looking at electric forces so it's sufficient to use Coulomb's law. It is possible to vectorize the code such that there are no loops. The solution shows a way to do it with one loop. NumPy contains many directories that are useful to vectorize code.


% --- begin solution of exercise ---
\paragraph{Solution.}
\begin{verbatim}
def force(self, r = None):
    """
    find electric force between particles
    """
    #testing if we gave r a value
    if r is None:
        r_test = self.r
    else:
        r_test = r
    #force array we are going to fill
    F = np.zeros((2, self.N))
    #possible to drop this for loop
    for i in range(self.N):
        #calculating the force between every particle
        r_eval = r_test[:,i]
        ri = np.tile(r_eval, self.N-1)
        ri = np.reshape(ri, (self.N-1, 2)).T
        rm = np.delete(r_test, i, axis=1)
        r_mi = ri - rm
        F_mi = r_mi/np.linalg.norm(r_mi, axis=0)**2
        F[:, i] = np.sum(F_mi, axis=1)
    #need to add the natural constants
    F *= self.q**2/(4*np.pi*self.eps0)
    return F
\end{verbatim}

% --- end solution of exercise ---

\subex{c)}
To make sure the forces makes sence it can be a good idea visualize the forces. Make a vector plot using \texttt{quiver} from \texttt{matplotlib.pyplot}. Create a function \texttt{plot\_arrows} that plots the force-arrows on each particle. In order to make the plot easier to understand plot the particles' positions and the borders of the box.

% --- begin hint in exercise ---

\paragraph{Hint.}
To plot points instead of lines you can use \texttt{plt.scatter}.

% --- end hint in exercise ---


% --- begin solution of exercise ---
\paragraph{Solution.}
\begin{verbatim}
def plot_border(self):
    """
    plot the box border
    """
    plt.plot([-self.L/2, self.L/2],
             [-self.L/2, -self.L/2], 'k')  # bottom side
    plt.plot([-self.L/2, self.L/2],
             [self.L/2, self.L/2], 'k')    # top side
    plt.plot([-self.L/2, -self.L/2],
             [-self.L/2, self.L/2], 'k')   # left side
    plt.plot([self.L/2, self.L/2],
             [-self.L/2, self.L/2], 'k',
             \label = 'Walls')              # right side
def plot_positions(self):
    """
    plot current positions as scatter-plot
    """
    plt.scatter(self.r[0,:], self.r[1,:], color='b', \label='Electrons')
def plot_arrows(self):
    """
    plot force arrows
    """
    F = self.force()
    plt.quiver(self.r[0,:], self.r[1,:], F[0, :], F[1, :], color='r', \label='Force')
\end{verbatim}

% --- end solution of exercise ---

\subex{d)}
Now we are ready to simulate the motion of the particles. First expand your \texttt{\_\_init\_\_} to take a small time step $\Delta t$ and the period $T$ you want to evaluate. Also define the number of steps $N_{steps}$ this will require.


% --- begin solution of exercise ---
\paragraph{Solution.}
Expanding on your previous \texttt{\_\_init\_\_} function it should look something like this (changes are marked with \texttt{\#!\#}):
\begin{verbatim}
def __init__(self, N, L, q, m, eps0, dt, T):
    """
    Set initial conditions
    """
    self.N = int(N)                     # num of particles
    self.L = L                          # length of side
    self.q = q                          # charge
    self.eps0 = eps0                    # permittivity of vacuum
    self.m = m                          # mass
    self.T = T                          # time we are looking at #!#
    self.dt = dt                        # time step              #!#
    self.Nsteps = int(T/dt)                                      #!#
    self.r = np.random.uniform(
        low=-L/2, high=L/2, size=(2, self.N)
    )                                   # initial positions
    self.v = np.random.normal(
        loc=0, scale=10, size=(2, self.N)
    )
\end{verbatim}

% --- end solution of exercise ---

\subex{e)}
To find the motion of the particles it is good practice to separate the function that solves the differential equations from the numeric solver that solves for time. To solve the differential equations we are simply going to write a function \texttt{RHS} that takes a vector $r_0 = (r, v)$ containing positions and velocities and returns the derivative of that vector $r' = (r', v') = (v, a)$. You can assume elastic collision with the walls.

% --- begin hint in exercise ---

\paragraph{Hint.}
To find the particles hitting the walls you can use \texttt{np.where()}.

% --- end hint in exercise ---


% --- begin solution of exercise ---
\paragraph{Solution.}
The code can look something like this:
\begin{verbatim}
def RHS(self, r0):
    """
    returns right hand side of the ode
    """
    #the array we will return
    drdt = np.zeros((2,2,self.N))
    #finding what particles are hitting the walls
    index = np.where(np.abs(r0[0, :])>self.L/2)
    #changing the sign of the velocity (elastic collision)
    r0[1][index] *= -1
    #placing the particles outside the box inside again
    np.where(r0[0, :] > self.L/2, r0[0, :], self.L/2)
    np.where(r0[0, :] < -self.L/2, r0[0, :], -self.L/2)
    #filling the array
    drdt[0, :, :] = r0[1, :]
    drdt[1, :, :] = self.force(r = r0[0, :])/self.m
    return drdt
\end{verbatim}

% --- end solution of exercise ---

\subex{f)}
Choosing what solver we are going to use is important and you should ponder a while on what you should prioritize. If the simulation is going on for a while should we prioritize energy-conservation? Write a function \texttt{solver} that solves and returns the motion of the particles. The solution solves the motion using Euler-Chromer's method, but you should try another one and compare.


% --- begin solution of exercise ---
\paragraph{Solution.}
\begin{verbatim}
def solver(self):
    """
    solves the motions of the particles with Euler-Chromer
    """
    #the solution we will fill
    sol = np.zeros((self.Nsteps, 2, 2, self.N))
    #setting initial conditions
    sol[0, 0, :, :] = self.r[:, :]
    sol[0, 1, :, :] = self.v[:, :]
    for i in range(self.Nsteps-1):
        drdt = self.RHS(sol[i, :, :, :])
        sol[i+1, :, :] = drdt*dt + sol[i, :, :, :]
    #the time we evaluated
    t = np.linspace(0, self.T, self.Nsteps)
    return sol, t
\end{verbatim}

% --- end solution of exercise ---

\subex{g)}
Finally animate the motion of the particles. To do this you can use \texttt{matplotlib.animate}.


% --- begin solution of exercise ---
\paragraph{Solution.}
\begin{verbatim}
#animation
sol, t = system.solver()
Nsteps = len(t)
#Creating the figure we are going to use
fig = plt.figure()
ax = plt.axes(xlim=(-L/2+0*1*L, L/2+0*1*L), ylim=(-L/2+0*1*L, L/2+0*1*L))
#the particles we are goint to plot
particles, = ax.plot([], [], 'bo', \label='Electrons')
#the walls
ax.plot([-L/2, L/2],
        [-L/2, -L/2], 'k')  # bottom side
ax.plot([-L/2, L/2],
        [L/2, L/2], 'k')    # top side
ax.plot([-L/2, -L/2],
        [-L/2, L/2], 'k')   # left side
ax.plot([L/2, L/2],
        [-L/2, L/2], 'k',
        \label = 'Walls')   # right side
def init():
    """
    init function that clears axis
    """
    ax.clear
    return particles
speed = 1
def animate(i):
    """
    this function animates
    """
    ax.set_title('Tid =%fs' %(i*dt*speed))
    particles.set_data(sol[i*speed, 0, :, :])
    return particles
#plot
anim = animation.FuncAnimation(fig, animate, init_func=init,
                               frames=int(Nsteps/speed), interval=1, blit=False)
plt.suptitle('Animation of particles', fontsize=14)
ax.set_y\label('Position y-axis [m]')
ax.set_x\label('Position x-axis [m]')
plt.axis('equal')
plt.legend(loc=1)
plt.show()
\end{verbatim}

% --- end solution of exercise ---

\end{doconceexercise}
% --- end exercise ---


\subsection{The entire code:}
\begin{verbatim}
import numpy as np
import matplotlib.pyplot as plt
import scipy.constants as const
from matplotlib import animation
from matplotlib import style
style.use('seaborn') #this is just visual
class box:
    def __init__(self, N, L, q, m, eps0, dt, T):
        """
        Set initial conditions
        """
        self.N = int(N)                     # num of particles
        self.L = L                          # length of side
        self.q = q                          # charge
        self.eps0 = eps0                    # permittivity of vacuum
        self.m = m                          # mass
        self.T = T                          # time we are looking at ###
        self.dt = dt                        # time step              ###
        self.Nsteps = int(T/dt)                                      ###
        self.r = np.random.uniform(
            low=-L/2, high=L/2, size=(2, self.N)
        )                                   # initial positions
        self.v = np.random.normal(
            loc=0, scale=10, size=(2, self.N)
        )                                   # initial velocities
    def plot_border(self):
        """
        plot the box border
        """
        plt.plot([-self.L/2, self.L/2],
                 [-self.L/2, -self.L/2], 'k')  # bottom side
        plt.plot([-self.L/2, self.L/2],
                 [self.L/2, self.L/2], 'k')    # top side
        plt.plot([-self.L/2, -self.L/2],
                 [-self.L/2, self.L/2], 'k')   # left side
        plt.plot([self.L/2, self.L/2],
                 [-self.L/2, self.L/2], 'k',
                 \label = 'Walls')              # right side
    def plot_positions(self):
        """
        plot current positions as scatter-plot
        """
        plt.scatter(self.r[0,:], self.r[1,:], color='b', \label='Electrons')
    def force(self, r = None):
        """
        find electric force between particles
        """
        #testing if we gave r a value
        if r is None:
            r_test = self.r
        else:
            r_test = r
        #force array we are going to fill
        F = np.zeros((2, self.N))
        #possible to drop this for loop
        for i in range(self.N):
            #calculating the force between every particle
            r_eval = r_test[:,i]
            ri = np.tile(r_eval, self.N-1)
            ri = np.reshape(ri, (self.N-1, 2)).T
            rm = np.delete(r_test, i, axis=1)
            r_mi = ri - rm
            F_mi = r_mi/np.linalg.norm(r_mi, axis=0)**2
            F[:, i] = np.sum(F_mi, axis=1)
        #need to add the natural constants
        F *= self.q**2/(4*np.pi*self.eps0)
        return F
    def plot_arrows(self):
        """
        plot force arrows
        """
        F = self.force()
        plt.quiver(self.r[0,:], self.r[1,:], F[0, :], F[1, :], color='r', \label='Force')
    def RHS(self, r0):
        """
        returns right hand side of the ode
        """
        #the array we will return
        drdt = np.zeros((2,2,self.N))
        #finding what particles are hitting the walls
        index = np.where(np.abs(r0[0, :])>self.L/2)
        #changing the sign of the velocity (elastic collision)
        r0[1][index] *= -1
        #placing the particles outside the box inside again
        np.where(r0[0, :] > self.L/2, r0[0, :], self.L/2)
        np.where(r0[0, :] < -self.L/2, r0[0, :], -self.L/2)
        #filling the array
        drdt[0, :, :] = r0[1, :]
        drdt[1, :, :] = self.force(r = r0[0, :])/self.m
        return drdt
    def solver(self):
        """
        solves the motions of the particles with Euler-Chromer
        """
        #the solution we will fill
        sol = np.zeros((self.Nsteps, 2, 2, self.N))
        #setting initial conditions
        sol[0, 0, :, :] = self.r[:, :]
        sol[0, 1, :, :] = self.v[:, :]
        for i in range(self.Nsteps-1):
            drdt = self.RHS(sol[i, :, :, :])
            sol[i+1, :, :] = drdt*dt + sol[i, :, :, :]
        #the time we evaluated
        t = np.linspace(0, self.T, self.Nsteps)
        return sol, t
#defining constants
q = -const.e
m_e = const.m_e
eps0 = const.epsilon_0
dt = 1e-4
T = 1
N = 50
L = 1
#creating our box of particles
system = box(N=N, L=L, q=q, m=m_e, eps0=eps0, dt=dt, T=T)
#running som plot functions we made
system.plot_border()
system.plot_arrows()
system.plot_positions()
system.force()
#plot
plt.x\label('Position x-axis [m]')
plt.y\label('Position y-axis [m]')
plt.axis('equal')
plt.title('Positions and the force exerted on the particles')
plt.legend(loc=1)
plt.show()
#animation
sol, t = system.solver()
Nsteps = len(t)
#Creating the figure we are going to use
fig = plt.figure()
ax = plt.axes(xlim=(-L/2+0*1*L, L/2+0*1*L), ylim=(-L/2+0*1*L, L/2+0*1*L))
#the particles we are goint to plot
particles, = ax.plot([], [], 'bo', \label='Electrons')
#the walls
ax.plot([-L/2, L/2],
        [-L/2, -L/2], 'k')  # bottom side
ax.plot([-L/2, L/2],
        [L/2, L/2], 'k')    # top side
ax.plot([-L/2, -L/2],
        [-L/2, L/2], 'k')   # left side
ax.plot([L/2, L/2],
        [-L/2, L/2], 'k',
        \label = 'Walls')   # right side
def init():
    """
    init function that clears axis
    """
    ax.clear
    return particles
speed = 1
def animate(i):
    """
    this function animates
    """
    ax.set_title('Tid =%fs' %(i*dt*speed))
    particles.set_data(sol[i*speed, 0, :, :])
    return particles
#plot
anim = animation.FuncAnimation(fig, animate, init_func=init,
                               frames=int(Nsteps/speed), interval=1, blit=False)
plt.suptitle('Animation of particles', fontsize=14)
ax.set_y\label('Position y-axis [m]')
ax.set_x\label('Position x-axis [m]')
plt.axis('equal')
plt.legend(loc=1)
plt.show()
\end{verbatim}

% ------------------- end of main content ---------------

% #ifdef PREAMBLE
\end{document}
% #endif

