%%
%% Automatically generated file from DocOnce source
%% (https://github.com/hplgit/doconce/)
%%
%%
% #ifdef PTEX2TEX_EXPLANATION
%%
%% The file follows the ptex2tex extended LaTeX format, see
%% ptex2tex: http://code.google.com/p/ptex2tex/
%%
%% Run
%%      ptex2tex myfile
%% or
%%      doconce ptex2tex myfile
%%
%% to turn myfile.p.tex into an ordinary LaTeX file myfile.tex.
%% (The ptex2tex program: http://code.google.com/p/ptex2tex)
%% Many preprocess options can be added to ptex2tex or doconce ptex2tex
%%
%%      ptex2tex -DMINTED myfile
%%      doconce ptex2tex myfile envir=minted
%%
%% ptex2tex will typeset code environments according to a global or local
%% .ptex2tex.cfg configure file. doconce ptex2tex will typeset code
%% according to options on the command line (just type doconce ptex2tex to
%% see examples). If doconce ptex2tex has envir=minted, it enables the
%% minted style without needing -DMINTED.
% #endif

% #define PREAMBLE

% #ifdef PREAMBLE
%-------------------- begin preamble ----------------------

\documentclass[%
oneside,                 % oneside: electronic viewing, twoside: printing
final,                   % draft: marks overfull hboxes, figures with paths
10pt]{article}

\listfiles               %  print all files needed to compile this document

\usepackage{relsize,makeidx,color,setspace,amsmath,amsfonts,amssymb}
\usepackage[table]{xcolor}
\usepackage{bm,ltablex,microtype}

\usepackage[pdftex]{graphicx}

\usepackage[T1]{fontenc}
%\usepackage[latin1]{inputenc}
\usepackage{ucs}
\usepackage[utf8x]{inputenc}

\usepackage{lmodern}         % Latin Modern fonts derived from Computer Modern

% Hyperlinks in PDF:
\definecolor{linkcolor}{rgb}{0,0,0.4}
\usepackage{hyperref}
\hypersetup{
    breaklinks=true,
    colorlinks=true,
    linkcolor=linkcolor,
    urlcolor=linkcolor,
    citecolor=black,
    filecolor=black,
    %filecolor=blue,
    pdfmenubar=true,
    pdftoolbar=true,
    bookmarksdepth=3   % Uncomment (and tweak) for PDF bookmarks with more levels than the TOC
    }
%\hyperbaseurl{}   % hyperlinks are relative to this root

\setcounter{tocdepth}{2}  % levels in table of contents

% prevent orhpans and widows
\clubpenalty = 10000
\widowpenalty = 10000

\newenvironment{doconceexercise}{}{}
\newcounter{doconceexercisecounter}


% ------ header in subexercises ------
%\newcommand{\subex}[1]{\paragraph{#1}}
%\newcommand{\subex}[1]{\par\vspace{1.7mm}\noindent{\bf #1}\ \ }
\makeatletter
% 1.5ex is the spacing above the header, 0.5em the spacing after subex title
\newcommand\subex{\@startsection{paragraph}{4}{\z@}%
                  {1.5ex\@plus1ex \@minus.2ex}%
                  {-0.5em}%
                  {\normalfont\normalsize\bfseries}}
\makeatother


% --- end of standard preamble for documents ---


% insert custom LaTeX commands...

\raggedbottom
\makeindex
\usepackage[totoc]{idxlayout}   % for index in the toc
\usepackage[nottoc]{tocbibind}  % for references/bibliography in the toc

%-------------------- end preamble ----------------------

\begin{document}

% matching end for #ifdef PREAMBLE
% #endif

\newcommand{\exercisesection}[1]{\subsection*{#1}}


% ------------------- main content ----------------------



% --- begin exercise ---
\begin{doconceexercise}
\refstepcounter{doconceexercisecounter}

\exercisesection{Exercise \thedoconceexercisecounter: Derive field of magnetic dipole}


\emph{(Made by: Sigurd Sørlie Rustad)}

\noindent
In this exercise we are going to derive the magnetic field from a magnetic dipole. First we are going  to consider the field from an electric dipole. Lets say we have two oppositely charged particles a distance $d$ from eachother. We place origin between the two particles and the particles are placed along the $z$-axis.


\subex{a)}
Find the electric field in $\mathbf{r} = (x, y, z)$, where $|\mathbf{r}|\gg d$.


% --- begin solution of exercise ---
\paragraph{Solution.}
We know the electric potential for a positive charge in origin to be
\begin{equation}
V(x, y, z) = \frac{Q}{4\pi \epsilon \sqrt{x² + y² + z²}}
\end{equation}
Using this we can write down the electric potential from the particles.
\begin{equation}
V(x, y, z) = \frac{1}{4\pi\epsilon}\left[\frac{q}{\sqrt{x² + y² + \big(z-d/2\big)²}} + \frac{-q}{\sqrt{x² + y² + \big(z+d/2\big)²}}\right]
\end{equation}
Because we are looking at the positions where $|\mathbf{r}|\gg d$ We can simplify by inserting $(z-d/2)² = z² + zd + d² \approx z²-zd$. This is called a binomial expansion.
\begin{align}
\frac{1}{\sqrt{x² + y² + \big(z-d/2\big)²}} \approx \frac{1}{\sqrt{z²-zd +x² + y²}} = \frac{1}{\sqrt{r² - zd}} = \frac{1}{r}\frac{1}{\sqrt{1-\frac{zd}{r²}}}
\end{align}
Using binomial expansion once again we can simplify the term further
\begin{equation}
\frac{1}{r}\frac{1}{\sqrt{1-\frac{zd}{r²}}} \approx \frac{1}{r}\frac{1}{\sqrt{\big(1-\frac{zd}{2r²}\big)²}} = \frac{1}{r}\frac{1}{1-\frac{zd}{2r²}}
\end{equation}
Lastly we can Taylor expansion the term to get
\begin{equation}
\frac{1}{r}\frac{1}{1-\frac{zd}{2r²}} \approx \frac{1}{r}\bigg(1 + \frac{zd}{2r²}\bigg)
\end{equation}
Doing this for both terms we get a easy term for the potential
\begin{equation}
V(\mathbf{r}) = \frac{1}{4\pi\epsilon_0}\frac{z}{|\mathbf{r}|³}qd
\end{equation}
Now to get that term to look like the ordinary electric dipole term, we need to insert some new variables. Notice that the electric dipole is defined as $\mathbf{p} = qd\mathbf{\hat{z}} = q\mathbf{d}$, where $\mathbf{d}$ is the vector pointing between the two particles. Also $z/|\mathbf{r}| = cos(\phi)$ where $\phi$ is the angle between the $y$-axis and $\mathbf{r}$. This gives us the term
\begin{equation}
\frac{1}{4\pi\epsilon_0}\frac{|\mathbf{p}|cos(\phi)}{|\mathbf{r}|²}
\end{equation}
Last but not least we can use the definition of the dot product to insert $|\mathbf{p}|cos(\phi) = \mathbf{p} \cdot (\mathbf{r}/|\mathbf{r}|)$, giving us the potential of a electric dipole where $|\mathbf{r}|\gg d$.
\begin{equation}
V(\mathbf{r}) = \frac{1}{4\pi\epsilon_0}\frac{\mathbf{p}\cdot\mathbf{r}}{|\mathbf{r}|³}
\end{equation}

% --- end solution of exercise ---

\subex{b)}
Use the potential to find the electric field $\mathbf{E}(\mathbf{r})$.


% --- begin solution of exercise ---
\paragraph{Solution.}
We need to solve the equation $\mathbf{E}=-\nabla V$. Then it is easier to work with the expression
\begin{equation}
V(\mathbf{r}) = \frac{|\mathbf{p}|}{4\pi\epsilon_0}\frac{z}{|\mathbf{r}|³}
\end{equation}
Working through the math, gives us the solution
\begin{equation}
E_x = \frac{|\mathbf{p}|}{4\pi\epsilon_0}\frac{3zx}{|\mathbf{r}|⁵}\mathbf{\hat{x}} \ \ \wedge \ \ E_y = \frac{|\mathbf{p}|}{4\pi\epsilon_0}\frac{3zy}{|\mathbf{r}|⁵}\mathbf{\hat{y}} \ \ \wedge \ \ E_z = \frac{|\mathbf{p}|}{4\pi\epsilon_0}\bigg(\frac{1}{|\mathbf{r}|³} - \frac{3z²}{|\mathbf{r}|⁵}\bigg)\mathbf{\hat{z}}
\end{equation}
Combining the terms we get our final term
\begin{equation}
\mathbf{E}(\mathbf{r}) = \frac{3(\mathbf{p}\cdot \mathbf{\hat{r}})\mathbf{\hat{r}} - \mathbf{p}}{4\pi\epsilon_0|\mathbf{r}|³}, \ \ \ \ \mathbf{\hat{r}} = \mathbf{r}/|\mathbf{r}|
\end{equation}

% --- end solution of exercise ---

\subex{c)}
It turns out that the magnetic field for a dipole has the same as the field for a electric dipole, if you do the necessary substitutions. Do the substitutions and find the magnetic field $\mathbf{B}$.


% --- begin solution of exercise ---
\paragraph{Solution.}
You only need to make two substitutions
\begin{equation}
\mathbf{p} = \frac{\bm{\mu}}{c²} \ \ \ \ \wedge \ \ \ \ \epsilon_0 = \frac{1}{\mu_0 c²}
\end{equation}
Where $\bm\mu$ is the magnetic momen and $c$ the speed of light. This gives us the final solution
\begin{equation}
\mathbf{B}(\mathbf{r}) = \frac{\mu_0}{4\pi}\frac{3(\bm{\mu}\cdot\mathbf{\hat{r}})\mathbf{\hat{r}} - \bm{\mu}}{|\mathbf{r}|³}
\end{equation}
It is also possible to find the field through calculating the field fram a charge following a small loop, but this requires a bit more work.

% --- end solution of exercise ---








\end{doconceexercise}
% --- end exercise ---


% ------------------- end of main content ---------------

% #ifdef PREAMBLE
\end{document}
% #endif

