%%
%% Automatically generated file from DocOnce source
%% (https://github.com/hplgit/doconce/)
%%
%%
% #ifdef PTEX2TEX_EXPLANATION
%%
%% The file follows the ptex2tex extended LaTeX format, see
%% ptex2tex: http://code.google.com/p/ptex2tex/
%%
%% Run
%%      ptex2tex myfile
%% or
%%      doconce ptex2tex myfile
%%
%% to turn myfile.p.tex into an ordinary LaTeX file myfile.tex.
%% (The ptex2tex program: http://code.google.com/p/ptex2tex)
%% Many preprocess options can be added to ptex2tex or doconce ptex2tex
%%
%%      ptex2tex -DMINTED myfile
%%      doconce ptex2tex myfile envir=minted
%%
%% ptex2tex will typeset code environments according to a global or local
%% .ptex2tex.cfg configure file. doconce ptex2tex will typeset code
%% according to options on the command line (just type doconce ptex2tex to
%% see examples). If doconce ptex2tex has envir=minted, it enables the
%% minted style without needing -DMINTED.
% #endif

% #define PREAMBLE

% #ifdef PREAMBLE
%-------------------- begin preamble ----------------------

\documentclass[%
oneside,                 % oneside: electronic viewing, twoside: printing
final,                   % draft: marks overfull hboxes, figures with paths
10pt]{article}

\listfiles               %  print all files needed to compile this document

\usepackage{relsize,makeidx,color,setspace,amsmath,amsfonts,amssymb}
\usepackage[table]{xcolor}
\usepackage{bm,ltablex,microtype}

\usepackage[pdftex]{graphicx}

\usepackage[T1]{fontenc}
%\usepackage[latin1]{inputenc}
\usepackage{ucs}
\usepackage[utf8x]{inputenc}

\usepackage{lmodern}         % Latin Modern fonts derived from Computer Modern

% Hyperlinks in PDF:
\definecolor{linkcolor}{rgb}{0,0,0.4}
\usepackage{hyperref}
\hypersetup{
    breaklinks=true,
    colorlinks=true,
    linkcolor=linkcolor,
    urlcolor=linkcolor,
    citecolor=black,
    filecolor=black,
    %filecolor=blue,
    pdfmenubar=true,
    pdftoolbar=true,
    bookmarksdepth=3   % Uncomment (and tweak) for PDF bookmarks with more levels than the TOC
    }
%\hyperbaseurl{}   % hyperlinks are relative to this root

\setcounter{tocdepth}{2}  % levels in table of contents

% prevent orhpans and widows
\clubpenalty = 10000
\widowpenalty = 10000

\newenvironment{doconceexercise}{}{}
\newcounter{doconceexercisecounter}


% ------ header in subexercises ------
%\newcommand{\subex}[1]{\paragraph{#1}}
%\newcommand{\subex}[1]{\par\vspace{1.7mm}\noindent{\bf #1}\ \ }
\makeatletter
% 1.5ex is the spacing above the header, 0.5em the spacing after subex title
\newcommand\subex{\@startsection{paragraph}{4}{\z@}%
                  {1.5ex\@plus1ex \@minus.2ex}%
                  {-0.5em}%
                  {\normalfont\normalsize\bfseries}}
\makeatother


% --- end of standard preamble for documents ---


% insert custom LaTeX commands...

\raggedbottom
\makeindex
\usepackage[totoc]{idxlayout}   % for index in the toc
\usepackage[nottoc]{tocbibind}  % for references/bibliography in the toc

%-------------------- end preamble ----------------------

\begin{document}

% matching end for #ifdef PREAMBLE
% #endif

\newcommand{\exercisesection}[1]{\subsection*{#1}}


% ------------------- main content ----------------------



% --- begin exercise ---
\begin{doconceexercise}
\refstepcounter{doconceexercisecounter}

\exercisesection{Exercise \thedoconceexercisecounter: Visualize electric field}


Here we are going to visualize the electric field from several particles in 2D. We are going to do this in parts.


\subex{a)}
Consider a particle in $(x_1, y_1)$ with charge $Q$, write down the general electric field in position $(x, y)$.


% --- begin solution of exercise ---
\paragraph{Solution.}
The vector from the particle and the position you want to evaluate is given by $\mathbf{R} = [x - x_1, y-y_1]$. We can then write the general electric field:
\begin{equation}
\mathbf{E} = \frac{Q}{4\pi\epsilon _0\big((x-x_1)^2 + (y-y_1)^2\big)^{3/2}}[x-x_1, y-y_1]
\end{equation}

% --- end solution of exercise ---

\subex{b)}
Make a program that takes in position and charge from a particle, a point in space, and returns the electric field vector in that point. Make a vector arrow plot.

% --- begin hint in exercise ---

\paragraph{Hint 1.}
It can be a good idea to scale $\epsilon _0$ so you get reasonable sizes.

% --- end hint in exercise ---

% --- begin hint in exercise ---

\paragraph{Hint 2.}
Use quiver from matplotlib.pyplot to visualize the field.

% --- end hint in exercise ---


% --- begin solution of exercise ---
\paragraph{Solution.}
\begin{verbatim}
import numpy as np
import matplotlib.pyplot as plt
eps0 = 1 #scaled to get reasonable sizes
def Efield(pos, Q, x, y):
    """
    This function takes in the position and charge of a particle, the position
    you want to calculate the field, and returns the x- and y-coordinate of
    the field.
    """
    r_eval = np.array([x-pos[0], y-pos[1]])
    R_eval = np.linalg.norm(r_eval, axis=0)
    Field = Q/(4*np.pi*eps0*R_eval**3)*r_eval
    return Field[0], Field[1]
#The area we are going to be looking at
x = np.linspace(-10,10,20)
y = x
X, Y = np.meshgrid(x,y)
#The position and charge of the particle
pos1 = np.array([1, 0])
Q1 = 1
#Calculating the electric field
U, V = Efield(pos1, Q1, X, Y)
#Plotting
fig, ax = plt.subplots()
ax.quiver(X, Y, U, V)
ax.plot(pos1[0], pos1[1], 'or')
plt.show()
\end{verbatim}

% --- end solution of exercise ---

\subex{c)}
Expand the function you made to take in an arbitrary number of particles and then returns the resulting electric field. Make a vector arrow plot.


% --- begin solution of exercise ---
\paragraph{Solution.}
\begin{verbatim}
eps0 = 1 #scaled to get reasonable sizes
def Efield_expanded(positions, charges, x, y):
    """
    This function takes in positions and charges, positions we want to evaluate,
    and returns the resulting electric field.
    It also returns a list with colors that corresponds to the charges of the particles.
    """
    Field = 0
    color = []
    for pos, Q in zip(positions, charges):
        r_eval = np.array([x-pos[0], y-pos[1]])
        R_eval = np.linalg.norm(r_eval, axis=0)
        Field += Q/(4*np.pi*eps0*R_eval**3)*r_eval
        if Q > 0:
            color.append('r')
        else:
            color.append('b')
    return Field[0], Field[1], color
#Defining positions and charges
pos1 = np.array([5, 5])
Q1 = 1
pos2 = np.array([0, 0])
Q2 = -1
pos3 = np.array([-2, -2])
Q3 = -3
positions = [pos1, pos2, pos3]
charges = [Q1, Q2, Q3]
#The area we are going to be looking at
x = np.linspace(-10,10,40)
y = x
X, Y = np.meshgrid(x, y)
#calculating the electric field
U, V, colors = Efield_expanded(positions, charges, X, Y)
#plotting
fig, ax = plt.subplots()
for col, pos in zip(colors, positions):
    ax.plot(pos[0], pos[1], 'o', color=col)
ax.quiver(X, Y, U, V)
plt.show()
\end{verbatim}

% --- end solution of exercise ---

\end{doconceexercise}
% --- end exercise ---


% ------------------- end of main content ---------------

% #ifdef PREAMBLE
\end{document}
% #endif

